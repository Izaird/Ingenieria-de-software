\documentclass[letterpaper,11pt]{article}

%%%%%%%   PAQUETES   %%%%%%%
\usepackage[spanish]{babel}
\usepackage{graphicx}
\usepackage[utf8]{inputenc}
\usepackage{wallpaper}
\usepackage[pdfstartview=FitH]{hyperref}
\usepackage{tabularx}
\usepackage{glossaries}
\usepackage{geometry}
%\usepackage[autostyle=true,german=quotes]{csquotes}
\usepackage[capitalise]{cleveref}
\usepackage{appendix}
\usepackage{pdfpages}



\begin{document}
%%%%%%%%%%%%%%%%%%%%%%%%%%%%
%%%%%      PORTADA      %%%%%
%%%%%%%%%%%%%%%%%%%%%%%%%%%%%

\begin{titlepage}


    \ThisLRCornerWallPaper{1}{imgs/fondo_tt.png} % Fondo de portada 
        \begin{center}
            \LARGE \textbf{Instituto Politécnico Nacional}\\*[0.3cm]
            \Large {Escuela Superior de Cómputo}\\
            \vspace{1cm}
            \Large {Planificación}\\
            \rule{12cm}{0.5mm}\\*[0.3cm]% Línea {Longitud}{Grosor}
            \hspace{0.9cm} 
            \normalsize {\textit{Ingeniería en Sistemas Computacionales}}\\
            \Large {\bf Ingeniería de Software}\\*[1cm]
            \large {Grupo: 3CV4 }\\*[0.5cm]
            \large {\today}\\*[1cm]
        \end{center}

    \centering %Todo centrado

    %%%%  LOGO DE LA ESCUELA   %%%%
    %\includegraphics[scale=0.17]{imagenes/escom-ipn} %Imagen para portada
    %%%%  NOMBRE DE LA ESCUELA   %%%%
    %\LARGE{\\ Instituto Polit\'ecnico Nacional}
    %\LARGE{\\ Escuela Superior de C\'omputo}
    
    %\vspace{1cm} %Espacio vertical

    %%%%  TITULO Y NÚMERO DE TRABAJO   %%%%
    %\LARGE \textbf{ Nombre del Trabajo Terminal}
    %\LARGE {\\ Número de Trabajo Terminal}

    %\vspace{1cm} %Espacio vertical

    %\LARGE \textit{Que para cumplir con la opción de titulación curricular en la carrera de:}
    %\LARGE \textbf{\\ Ingeniería en Sistemas Computacionales}

    \vspace{1cm} %Espacio vertical

    %%%%   ALUMNOS   %%%%
   \textit{Equipo 4 :}\\ 
     Pineda Vieyra Itzcoatl Rodrigo \\
     Nuncio Pulido Eric Emmanuel\\
     Mothelet Delgado Izaird Alexander\\
      Melo Barranco Maria Celeste\\
      Rosales Valdez Edna Nemesis



    \vspace{1cm} %Espacio vertical


\end{titlepage}

\section{Análisis}
\subsection{Objetivos}
Los objetivos de la primera iteración son
\begin{enumerate}
    \item Elaborar un sistema que le permita al Gobierno de La Ciudad de México recibir la información necesaria para la atención de baches.
    \item Notificar al Gobierno de la Ciudad de México por medio del sistema.
\end{enumerate}
\subsection{Alternativas}
Se contemplaron las siguientes soluciones al problema, dado el tiempo limite:
\begin{enumerate}
    \item Mapa interactivo de la CDMX, que marque la ubicación del bache.
    \item Formulario con la ubicación del bache y 4 fotos máximo de tamaño y formato especifico.
    \begin{enumerate}
        \item Bot de Twitter que tome la información del formulario y notifique a las autoridades correspondientes.
        \item Almacenar los datos en una base de datos o JSON para su posterior envió a las autoridades correspondientes.
    \end{enumerate}
\end{enumerate}
La solución elegida fue la 2.b) dado que es la de menor riesgo y la que mas rápido se puede implementar por el equipo de trabajo.
\subsection{Riegos}
Los principales riesgos son:
\begin{enumerate}
    \item Sobre-analizar y no dejar tiempo para el desarrollo y pruebas.
    \item Entrada al mercado de alternativas.
\end{enumerate}
\section{Diseño}
\subsection{Requerimientos}
\subsubsection{Requerimientos del usuario} 
\begin{enumerate}
    \item Los usuarios generales deberán poder llenar un formulario con la ubicación del bache.
    \item Los usuarios generales podrán subir hasta 4 fotos.
    \item Los usuarios de cuadrilla tendrán un formulario distinto, donde podrán dar más detalles.
\end{enumerate}
\subsubsection{Requisitos del sistema}
Requisitos Funcionales:
\begin{enumerate}
    \item El sistema deberá dar a conocer la ubicación del bache.
    \item Se podrá subir al menos una foto del bache.
    \item El sistema contará con un formulario distinto para cuadrillas de bacheo de la CDMX.
\end{enumerate}

Requisitos No Funcionales:
\begin{enumerate}
    \item El sistema podrá visualizarse en navegadores web (Chrome, Edge, Firefox, Safari, Opera).
    \item El sistema tendrá una interfaz optimizada para dispositivos móviles.
\end{enumerate}
\end{document}